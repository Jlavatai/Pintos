\documentclass[a4wide, 11pt]{article}
\usepackage{a4, fullpage}
\newcommand{\tab}{\hspace*{2em}}
\usepackage[margin=2cm]{geometry}
\newcommand{\tx}{\texttt}

\begin{document}

\title{OS211 \\ Task1: Threads \\ Design Document}
\author{Francesco Di Mauro, Thomas Rooney, Alex Rozanski}
\maketitle


\section{Alarm Clock}
\subsection{Data Structures}

Added to enum \texttt{thread\_status}: \\
\tab \tab \texttt{THREAD\_SLEEP} \\
\tab \tab Denotes a sleeping thread.
\\\\
Added to \texttt{struct\_thread}: \\
\tab\tab \texttt{vlong long wakeup\_tick} \\
\tab\tab If a thread is is sleeping, this is the tick it's going to be woken up on.
\\\\
Added to \texttt{thread.c} as a global variable: \\
\tab\tab \texttt{static struct list sleeping\_list} \\
\tab\tab List of threads currently sleeping ordered by the tick for the thread to wake up on.

\subsection{Algorithms}
\subsubsection{A2}
The first operation that \texttt{timer\_sleep()} carries out is disabling interrupts. It then calls out to the new method \texttt{thread\_sleep()}, which calculates the tick that the thread can be woken up from, sets the status of the thread to \texttt{THREAD\_SLEEP}, then adds it to the list of sleeping threads. Finally, \texttt{thread\_sleep()} calls \texttt{schedule()} in order to schedule a different thread. Note that we don't yield here (using \texttt{thread\_yield()}) as that would add the thread to the ready list, which we don't want.

When a sleeping thread returns from \texttt{thread\_sleep()} after being woken up, interrupts will be restored to their old level, which restores the correct state.

\subsubsection{A3}

The list of sleeping threads is ordered by the \tx{wakeup\_tick} member
which has been added to \tx{struct thread}. This is an absolute tick value that the thread 
should sleep until, which is set when \tx{timer\_sleep()} is invoked. Thanks to this ordering, when we iterate over the sleeping threads in \tx{thread\_sleep\_ticker()} (which is called from the
timer interrupt handler) we can stop iteration as soon as a thread whose \tx{wakeup\_tick} value is no later than the current tick is found.

Furthermore, because the tick which we want the thread to wake up on is stored as
an absolute value rather than a relative number of ticks to sleep for, we don't
need to update any sleep state in the timer interrupt handler for the sleeping
threads.

\emph{Add example with values?}

\subsection{Synchronization}
\subsubsection{A4}
The function \tx{timer\_sleep()} disables interrupts before calling \tx{thread\_sleep()} (which is the critical section where we modify the sleeping list and then schedule 
another thread). Since interrupts are disabled for this small section, we won't 
be pre-empted by another thread, so sleeping a thread and then scheduling 
another is an atomic operation.

After making the current thread sleep, another thread is run by calling \tx{schedule()}. Therefore, although we could use a synchronisation primitive like a semaphore to enforce invocation of \tx{thread\_sleep()} as a critical section, \tx{schedule()} asserts that interrupts are disabled, so it makes the most sense to disable interrupts for this critical section as they need to be off for \tx{schedule()} anyway.

\subsubsection{A5}
As we disable interrupts before calling \tx{thread\_sleep()} from \tx{timer\_sleep()}, this ensures that timer interrupts aren't handled during the invocation of this function. This operation prevents race conditions occurring between multiple threads where
we may wake up sleeping threads and modify the sleep list.

\section{Rationale}
This design was preferred because allows the system to perform a fairly low amount of processing in the timer interrupt handler. As tx{thread\_sleep\_ticker()} is called every tick, this feature is crucial. Our initial design used an unordered list of sleeping threads, and each thread stored the number of ticks to sleep for. In the timer interrupt handler we would decrement the number of ticks each thread in the sleeping queue was
sleeping for, and then wake up the thread when this count hit zero. This implementation was really inefficient and time consuming because at every tick we were not just iterating over every sleeping thread , but also modifying the state of every thread.
\newpage


\section{Priority Scheduling}
\subsection{Data Structures}

Added to \tx{struct thread}: \\\\
\tab\tab \tx{struct list lock\_list}:\\
\tab\tab This is an ordered list of the thread's held locks, with highest priority first. \\
\tab\tab They are only added when they have also been donated to.
\\\\
 \tab\tab   \tx{int priority}:\\
 \tab\tab This contains the thread's default, non-donated-to, priority.
\\\\
\tab\tab \tx{struct lock *blocker}:\\
\tab\tab This is a pointer to the lock that is currently blocking it, or NULL \\
\tab\tab should the thread not be currently blocked by a lock.
    \\\\    
Added to \tx{struct lock}:
\\\\
\tab\tab bool \tx{donated\_flag}:\\
\tab\tab This is a flag indicating whether the lock has donated its priority to its holder, or not.
\\\\
\tab\tab \tx{struct list\_elem elem}:\\
\tab\tab This is a \tx{list\_elem} structure such that the lock can be tracked via the pintos list structure -\\
\tab\tab meaning a thread can know which locks it currently holds.
\\\\
Added to \tx{struct semaphore}:
  \\\\  
\tab\tab \tx{int *priority}:\\
\tab\tab This is a pointer to a thread's priority value, this is initially the thread that holds it,\\
\tab\tab but is updated to the greatest priority of the thread's currently in it's wait queue.
\newpage

\subsubsection{B2}
\begin{verbatim}
                                +----------------+
                                - ASCII Diagram: -
  Initial Setup:                +----------------+
         +-----------------------+           +--------------------------------+
  Locks: |     A     |     B     |  Threads: |    L     |    M     |    H     |
         |pri:NULL   |pri:NULL   |           |pri: 1    |pri: 2    |pri: 3    |
         |holder:NULL|holder:NULL|           |locks:[]  |locks:[]  |locks:[]  |
         +-----------------------+           |blocker:0 |blocker:0 |blocker:0 |
                                             +--------------------------------+
-thread_get_priority(L)=1 - thread_get_priority(M)=2 - thread_get_priority(H)=3-
================================================================================
  Thread L: lock_acquire(A);      Thread M: lock_acquire(b) 
         +-----------------------+           +--------------------------------+
  Locks: |     A     |     B     |  Threads: |    L     |    M     |    H     |
         |pri: L->pri|pri: M->pri|           |pri: 1    |pri: 2    |pri: 3    |
         |holder: L  |holder: M  |           |locks:[]  |locks:[]  |locks:[]  |
         -------------------------           |blocker:0 |blocker:0 |blocker:0 |
                                             +--------------------------------+
-thread_get_priority(L)=1 - thread_get_priority(M)=2 - thread_get_priority(H)=3-
================================================================================
  Thread M: lock_acquire(a) 
         +-----------------------+           +--------------------------------+
  Locks: |     A     |     B     |  Threads: |    L     |M-blocked |    H     |
         |pri: M->pri|pri: M->pri|           |pri: 1    |pri: 2    |pri: 3    |
         |holder: L  |holder: M  |           |locks:[A] |locks:[]  |locks:[]  |
         +-----------------------+           |blocker:0 |blocker:&L|blocker:0 |
                                             +--------------------------------+
-thread_get_priority(L)=2 - thread_get_priority(M)=2 - thread_get_priority(H)=3-
================================================================================
  Thread H: lock_acquire(b)
         +-----------------------+           +--------------------------------+
  Locks: |     A     |     B     |  Threads: |    L     |M-blocked |H-blocked |
         |pri: H->pri|pri: H->pri|           |pri: 1    |pri: 2    |pri: 3    |
         |holder: L  |holder: M  |           |locks:[A] |locks:[B] |locks:[]  |
         +-----------------------+           |blocker:0 |blocker:&L|blocker:&M|
                                             +--------------------------------+
-thread_get_priority(L)=3 - thread_get_priority(M)=3 - thread_get_priority(H)=3-
================================================================================
  Thread L: lock_release(a) -- about to unblock Thread M: lock_acquire(a) 
         +-----------------------+           +--------------------------------+
  Locks: |     A     |     B     |  Threads: |    L     |M-blocked |H-blocked |
         |pri: NULL  |pri: H->pri|           |pri: 1    |pri: 2    |pri: 3    |
         |holder:NULL|holder: M  |           |locks:[]  |locks:[B] |locks:[]  |
         +-----------------------+           |blocker:0 |blocker:&L|blocker:&M|
                                             +--------------------------------+
-thread_get_priority(L)=3 - thread_get_priority(M)=3 - thread_get_priority(H)=3-
================================================================================
   Thread M: lock_acquire(a) completes
         +-----------------------+           +--------------------------------+
  Locks: |     A     |     B     |  Threads: |    L     |    M     |H-blocked |
         |pri: M->pri|pri: H->pri|           |pri: 1    |pri: 2    |pri: 3    |
         |holder: M  |holder: M  |           |locks:[]  |locks:[B] |locks:[]  |
         +-----------------------+           |blocker:0 |blocker:0 |blocker:&M|
                                             +--------------------------------+
-thread_get_priority(L)=1 - thread_get_priority(M)=3 - thread_get_priority(H)=3-

--And so on and so forth -------------------------------------------------------
\end{verbatim}

\subsection{Algorithms}
\subsubsection{B3}
The waiters list for a semaphore is ordered by descending priority. When choosing a new thread to run, the first thread in the list is picked. Thus the highest priority thread wakes first.
\subsubsection{B4}
Consider the case where there is a data structure as below:

\begin{verbatim}
         +-----------------------+           +--------------------------------+
  Locks: |     A     |     B     |  Threads: |    L     |M-blocked |    H     |
         |pri: M->pri|pri: M->pri|           |pri: 1    |pri: 2    |pri: 3    |
         |holder: L  |holder: M  |           |locks:[A] |locks:[]  |locks:[]  |
         +-----------------------+           |blocker:0 |blocker:&L|blocker:0 |
                                             +--------------------------------+
-thread_get_priority(L)=2 - thread_get_priority(M)=2 - thread_get_priority(H)=3-
\end{verbatim}

Should High priority thread (H) call lock\_acquire(B), which is held by thread M, the sequence of events is such:
\begin{verbatim}
  (1) lock_available(B) returns false
  (2) The priorities of Thread M and thread H are compared
  (3) Since the priority of thread H is greater the priority of thread M
    (4) lock B's priority pointer is set to point to thread H
    (5) Since lock B hasn't been donated to thread M before:
      (6) lock B is donated to thread M (Function call) -->.
        (7) lock B is inserted into thread M's lock_list
        (8) Since thread M is blocked, and the blocker has lower priority than
        |   lock B's priority pointer
        | (9) Thread M's blocker's priority pointer is updated to lock B's 
        |     priority pointer.
        └---(Recursion Step on M = M->blocker->holder)
\end{verbatim}

At the end of this logical sequence, the data structure would be so:

\begin{verbatim}
         +-----------------------+           +--------------------------------+
  Locks: |     A     |     B     |  Threads: |    L     |M-blocked |H-blocked |
         |pri: H->pri|pri: H->pri|           |pri: 1    |pri: 2    |pri: 3    |
         |holder: L  |holder: M  |           |locks:[A] |locks:[B] |locks:[]  |
         +-----------------------+           |blocker:0 |blocker:&L|blocker:&M|
                                             +--------------------------------+
-thread_get_priority(L)=3 - thread_get_priority(M)=3 - thread_get_priority(H)=3-
\end{verbatim}

Nested donation is handled via the recursive step.

\subsubsection{B5}

Considering the case where data structure is as such:

\begin{verbatim}
--------------------------------------------------------------------------------
- Locks: |     A     |     B     |- Threads: |    L     |M-blocked |H-blocked |-
---------|pri: H->pri|pri: H->pri|-----------|pri: 1    |pri: 2    |pri: 3    |-
---------|holder: L  |holder: M  |-----------|locks:[A] |locks:[B] |locks:[]  |-
---------------------------------------------|blocker:0 |blocker:&L|blocker:&M|-
                                              ---------------------------------
-thread_get_priority(L)=3 - thread_get_priority(M)=3 - thread_get_priority(H)=3-
\end{verbatim}

Should lock\_release(A), be called by Thread L:
\begin{verbatim}
 (1) Since A's priority has been been donated to L
    (2) thread_restore_priority_lock(A) is called
      (3) list_remove(A->elem) is called to remove A from L's lock_list
  (4) The lock's priority is set to NULL
  (5) The lock's holder is set to NULL
  (6) sema_up(A->semaphore) is called
    (7)  The list of waiters is ensured sorted by descending priority
    (8)  The top priority waiter is popped off the list
    (9)  This waiter is unblocked
\end{verbatim}

\subsubsection{B6}
Consider an implementation where the \tx{struct thread} has two integers representing the base priority and the donated priority. The function \tx{set\_priority} could recompute the base priority or set the base priority and recompute the donated priority to be the max of the two values and yield. Suppose that another thread donates its own priority while this re-computation is taking place. This thread may read the old value of donated priority, and write to the new value such that the donated priority is lower than the base priority as shown in the diagram below:
\begin{verbatim}

        Thread A        set_pri(7)
        pri: 2     | pri:7  | pri: 7           | pri: 7
        dpri: 5    | dpri:5 | dpri: max(5,7)=7 | dpri: 7

        But Consider: Thread B donating to thread A whilst set_pri is computing: 

        Thread A        set_pri(7)
        pri: 2     | pri:7  | pri: 7           | pri: 7
        dpri: 5    | dpri:5 | dpri: max(5,7)=7 | dpri: 6
                      ^                        ^
                      | Read 5                 | Write 6
                      + 5 < 6 -----------------+
        Thread B     donate(A)
        pri: 6
        dpri: 6 
\end{verbatim}    

The way this potential race condition is avoided in our implementation is via only storing one integer per thread to represent the priority. Each lock points to the priority of the thread that holds it, and thus any updates via set priority automatically set the priorities of locks in a single instruction. Thus interrupts (which happen between processor instructions) can not cause a problem.

\section{Advanced Scheduler}
\subsection{Data Structures}
\subsubsection{C1}\tx{int *priority}

Added to \tx{struct thread}: \\
\tab\tab \tx{int recent\_cpu;} \\
\tab\tab An exponentially weighted moving average of the CPU time received by each thread.
\\\\
Added to \tx{thread.c}: \\
\tab\tab \tx{\#define MLFQS\_RECOMPUTE\_INTERVAL 4} \\
\tab\tab Amount of clock ticks after which the priorities of the threads will be recomputed.
\\\\
\tab\tab \tx{static struct list thread\_mlfqs\_queue}\\
\tab\tab Queue of the ready to run threads used by the mlfqs scheduler.
\\\\
\tab\tab \tx{static long long mlfqs\_recompute\_ticks} \\
\tab\tab Number of ticks until the thread priorities will be recomputed.
\\\\
\tab\tab \tx{static int mlfqs\_load\_avg}\\
\tab\tab The system's load average, an estimate of the number of threads ready to be run in the \\
\tab\tab past minute.
\newpage
\subsubsection{C2}
\begin{verbatim}
timer  recent_cpu    priority   thread
ticks   A   B   C   A   B   C   to run
-----  --  --  --  --  --  --   ------
 0      0   0   0  63  61  59      A
 4      4   0   0  62  61  59      A  
 8      8   0   0  61  61  59      B
12      8   4   0  61  60  59      A
16      12  4   0  60  60  59      B
20      12  8   0  60  59  59      A     
24      16  8   0  59  59  59      C
28      16  8   4  59  59  58      B 
32      16  12  4  59  58  58      A        
36      20  12  4  58  58  58      C
\end{verbatim}

\subsubsection{C3}

The specification does not clearly express whether the calculation of the priority for each thread is carried out before or after updating the \tx{recent\_cpu} value. While fulfilling the table, we assumed that \tx{recent\_cpu} is calculated first, and then the priority is updated. Our implementation of the advances scheduler matches this behaviour.  

\subsubsection{C4}

\subsubsection{C5}

\subsubsection{C6}

Our initial fixed-point implementation consisted of a struct containing a number used for the fixed-point number representation, and several functions to perform the necessary fixed-point operations using this struct. Even if the implementation was correct, we deemed it inefficient: calling a function to perform simple mathematical operations was a significant overhead, especially considering that the kernel needs to recompute the priority of all the threads every second. In the end, we created a \tx{fixed\_point} typedef to an \tx{int32\_t}, and then defined a series of inline functions which actually implement the methods needed to perform the fixed-point operations. We chose inline functions over macros because inline functions are more practical when dealing with adding assertions.

\end{document}