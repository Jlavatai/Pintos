\documentclass[a4wide, 11pt]{article}
\usepackage{a4, fullpage}
\newcommand{\tab}{\hspace*{2em}}
\newcommand{\tx}{\texttt}

\begin{document}

\title{OS211 \\ Task1: Threads \\ Design Document}
\author{Francesco Di Mauro, Thomas Rooney, Alex Rozanski}
\maketitle


\section{Alarm Clock}
\subsection{Data Structures}

Added to enum \texttt{thread\_status}: \\
\tab \tab \texttt{THREAD\_SLEEP} \\ \\
\tab \tab Member of the \texttt{enum thread\_status}, representing a sleeping thread.
\\\\
Added to \texttt{struct\_thread}: \\
\tab\tab \texttt{vlong long wakeup\_tick} \\
\tab\tab If a thread is is sleeping, this the tick it's going to be woken up on.
\\\\
Added to \texttt{thread.c} as a global variable: \\
\tab\tab \texttt{static struct list sleeping\_list} \\
\tab\tab Ordered list of processes currently sleeping.\\ 
\tab\tab Processes in this list 
   have state set to \texttt{THREAD\_SLEEP}. \\
\tab\tab This list is ordered suck that the head
   is the next thread to be woken up.

\subsection{Algorithms}
\subsubsection{A2}
The first operation that the function \texttt{timer\_sleep} carries out is disabling the interrupts, then it calls out to the new method \texttt{thread\_sleep()}. This function calculates the tick the thread needs to be woken up on, sets the status of the thread to \texttt{THREAD\_SLEEP}, then adds it to the sleeping list. Finally, \texttt{thread\_sleep()} will call \texttt{schedule}, in order to choose a different thread to run. When returning from \texttt{thread\_sleep()}, the interrupts will be re-enabled. Putting a thread to sleep should be an atomic operation because it accesses the kernel's thread structure: if an interrupt occurs while writing to the sleeping threads list, this will cause a fatal exception. 

\subsubsection{A3}

The list of sleeping threads is ordered by the \tx{wakeup\_tick} member
which has been added to \tx{struct thread}. This is an absolute tick value that the thread 
should sleep until, which is set when \tx{timer\_sleep()} is invoked. Thanks to this ordering, when we iterate over the sleeping threads in \tx{thread\_sleep\_ticker()} (which is called from the
timer interrupt handler) we can stop iteration as soon as a thread whose
\tx{wakeup\_tick} value is later than the current tick is found.
\\
Furthermore, because the tick which we want the thread to wake up on is stored as
an absolute value rather than a relative number of ticks to sleep for, we don't
need to update any sleep state in the timer interrupt handler for the sleeping
threads.

\emph{Add example with values?}

\subsection{Synchronization}

We chose this design because allows the system to perform a fairly low amount of processing in the
timer interrupt handler. As tx{thread\_sleep\_ticker()} is called every tick, this feature is crucial. Our
initial design used an unordered list of sleeping threads, and each thread
stored the number of ticks to sleep for. In the timer interrupt handler we
would decrement the number of ticks each thread in the sleeping queue was
sleeping for, and then wake up the thread when this count hit zero. This implementation was
really inefficient because at every tick we were not just iterating over every sleeping
thread , but also modifying the state of every thread.

\section{Advance Scheduler}
\subsection{Data Structures}

Added to \tx{struct thread}: \\
\tab\tab \tx{int recent\_cpu;} \\
\tab\tab An exponentially weighted moving average of the CPU time received by each thread.
\\\\
Added to \tx{thread.c}: \\
\tab\tab \tx{\#define MLFQS\_RECOMPUTE\_INTERVAL 4} \\
\tab\tab Amount of clock ticks after which the priorities of the threads will be recomputed.
\\\\
\tab\tab \tx{static long long mlfqs\_recompute\_ticks} \\
\tab\tab Number of ticks until the thread priorities will be recomputed.
\\\\
\tab\tab \tx{static int mlfqs\_load\_avg}\\
\tab\tab The system's load average, an estimate of the number of threads ready to be run in the \\
\tab\tab past minute.
\end{document}